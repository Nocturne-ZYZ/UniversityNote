\documentclass[UTF8]{article}
\usepackage{ctex}
\usepackage{amsmath}
\usepackage{graphicx}
\begin{document}
\section{第四讲\;\;}
\subsection{功\;\;功能定理}
\subsubsection{功}

    (1)物体作直线运动,恒力做功
    \[A = Fcos\theta \cdot \lvert \Delta \vec{r} \vert\;\;\;\;\;A = \vec{F}\cdot\vec{\Delta r}\]

    (2)物体作曲线运动,变力做功
    \[\mbox{元功:}dA = Fcos\theta\cdot\lvert d\vec{r}\vert = \vec{F}\cdot d\vec{r}\]

    \[\mbox{总功:}A = \int_A^BdA = \int_A^B\vec{F}\cdot d\vec{r} = \int_A^BFcos\theta \lvert d\vec{r}\vert\]

    (3)质点同时受几个力作用时

    力的叠加原理:$\vec{F} = \vec{F_1} + \vec{F_2} + \dots + \vec{F_N}$

    \;\;\;\;\;\;\;$A = A_{1AB} + A_{2AB} + \dots + A_{NAB}$

    说明:

    \;\;(1)合力的功等于各分力沿同一路径所做功的代数和

    \;\;(2)计算力对物体做功时,必须说明是哪个力对物体沿哪条路径所做的功

\subsubsection{动能定理}

    1、定义:质点动能\[E_k = \frac{1}{2}mv^2\mbox{ 或者 }E_k = \frac{p^2}{2m}\]

    2、质点的动能定理\[A_{\mbox{合}AB} = E_{kB} - E_{kA} = \Delta E_k = \frac{1}{2}mv_B^2 - \frac{1}{2}mv_A^2\]

    \;\;合外力对质点所做的功(其它物体对它所做的总功)等于质点动能的增量

    3、质点系动能定理

    对n个质点组成的质点系:对每个质点分别使用动能定理
    \[\sum_{i=1}^n A_{i\mbox{外}} + \sum_{i=1}^n A_{i\mbox{内} = \sum_{i=1}^n E_{kiB} - \sum_{i=1}^n E_{kiA}}\]

    所有外力对质点系做的功和内力对质点系做的功之和等于质点系总动能的增量

    注意:内力能改变系统的总动能,但不能改变系统的总动量
\subsection{保守力与非保守力}
\subsection{势能\;\;势能曲线}
\subsection{功能原理以及机械能守恒定律}
\end{document}