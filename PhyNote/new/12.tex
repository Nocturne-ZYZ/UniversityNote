\documentclass[UTF8]{article}
\usepackage{amsmath}
\usepackage{graphicx}
\usepackage{ctex}

\begin{document}
\section{第六讲\;\;流体力学}
\subsection{压强与平衡方程}

    物质的三态:固态,液态,气态

    流体(液态,气态):具有一定体积、无固定形状、易于变形,具有一定的流动性

    液体和气体的不同点:液体有一定体积,几乎不可压缩,黏性大;气体没有一定体积,充满整个容器,易压缩,粘性小

    连续介质假设:流体在其存在的空间是连续、无间隙分布的,可取微分元

    流体元:宏观足够小而微观足够大,流体物理量是大量流元的相应物理量的统计平均

\subsubsection{流体静力学压强}

    静止流体没有抵抗剪切形变的能力,作用在流体内任一面元上的应力必与该面元垂直

    在静止流体中任取一个小面元$d\vec{s}$,作用在此面元上的力为$d\vec{f}$

    \[\mbox{通常流体内部的压力:}d\vec{f} = -pd\vec{s}\]

    p称为流体静力学压强,p为标量,单位帕(Pa)

    流体中静压强与面元取向无关

\subsubsection{静止流体的平衡方程}

    作用在流元上的力可以分为两类

    面积力:可用压强表述,作用在流元外表面上

    体积力:作用在每一质量微元上,亦称质量力

    设单位质量流体上的体积力为$\vec{F}$,则$d\vec{F} = \rho \vec{F}dxdydz$

    对静止流体:$\rho \vec{F}dxdydz - \bigtriangledown pdxdydz = 0$

    即\[\rho \vec{F}\bigtriangledown p\;\;\;\rho F_x = \frac{\partial p}{\partial}\;\;\;\rho F_y = \frac{\partial p}{\partial y}\;\;\;\rho F_y = \frac{\partial p}{\partial }\]

    结论:体积力与压强梯度方向平行,体积力与等压面垂直

    重力场中的静止流体\[\frac{\partial p}{\partial x} = 0\;\;\;\frac{\partial p}{\partial y} = 0\;\;\;\frac{\partial p}{\partial z} = -\rho g\]

    设深度$z = z_A$处的压强$p_A$,$z = z_B$处的压强$p_B$,若密度为常量
    \[p_B = p_A - \int_{z_A}^{z_B}\rho gdz = p_A - \rho g(z_B - z_A)\]

    静止在重力场中的同种流体

    \;\;(1)液体中压强随距液面深度线性变化

    \;\;(2)等压面是水平面,与重力方向垂直

    \[\frac{dp}{dz} = -\rho g\;\;\;\;\;z + \frac{p}{\rho g} = c\mbox{(常数)}\]
\subsection{流体连续性原理}
\subsection{伯努利方程及其应用}
\subsection{粘滞流体的运动}
\subsection{运动物体在流体中的受力}
\end{document}