\documentclass[12pt,a4paper,twoside,openany,fleqn]{report}

\usepackage{CJK}

\usepackage{graphicx} % 加载图的包

% 重定义字体命令
\newcommand{\song}{\CJKfamily{song}} % 宋体 (Windows自带simsun.ttf)
\newcommand{\fs}{\CJKfamily{fs}} % 仿宋体 (华天字库htfs.ttf)
\newcommand{\kai}{\CJKfamily{kai}} % 楷体 (华天字库htkai.ttf)
\newcommand{\hei}{\CJKfamily{hei}} % 黑体 (Windows自带simhei.ttf)
\newcommand{\li}{\CJKfamily{li}} % 隶书 (Windows自带simli.ttf)
\newcommand{\you}{\CJKfamily{you}} % 幼圆体 (Windows自带simyou.ttf)
%%% 以上六种字体均为标准 GBK 字体, 包括 GBK 繁体字和一些不常用字, 推荐!!!

\newcommand{\xs}{\CJKfamily{xs}}
\newcommand{\shu}{\CJKfamily{shu}} % 舒体 (方正字库fzstk.ttf)
% \newcommand{\yourcommand}[参数个数]{内容} [参数个数]这个是可选的。
% 例如 \newcommand{\you}{\CJKfamily{you}} 用\you 来代替 \CJKfamily{you} ,少输入很多字哦。

%------------------------------------------------------------------------
%字号设置
\newcommand{\chuhao}{\fontsize{42pt}{\baselineskip}\selectfont}
\newcommand{\xiaochuhao}{\fontsize{36pt}{\baselineskip}\selectfont}
\newcommand{\yihao}{\fontsize{28pt}{\baselineskip}\selectfont}
\newcommand{\xiaoyihao}{\fontsize{24pt}{\baselineskip}\selectfont}
\newcommand{\erhao}{\fontsize{21pt}{\baselineskip}\selectfont}
\newcommand{\xiaoerhao}{\fontsize{18pt}{\baselineskip}\selectfont}
\newcommand{\sanhao}{\fontsize{15.75pt}{\baselineskip}\selectfont}
\newcommand{\sihao}{\fontsize{14pt}{\baselineskip}\selectfont}
\newcommand{\xiaosihao}{\fontsize{12pt}{\baselineskip}\selectfont}
\newcommand{\wuhao}{\fontsize{10.5pt}{\baselineskip}\selectfont}
\newcommand{\xiaowuhao}{\fontsize{9pt}{\baselineskip}\selectfont}
\newcommand{\liuhao}{\fontsize{7.875pt}{\baselineskip}\selectfont}
\newcommand{\qihao}{\fontsize{5.25pt}{\baselineskip}\selectfont}
% \baselineskip | distance from bottom of one line of a paragraph to bottom of the next line. 基本行距
% 只有使用\selectfont命令之后,\fontzize{}{}的设置才能生效
% 可以用数字表示{11pt}:单倍行距


\begin{document}
\begin{CJK*}{GBK}{song}

%%%%%%%%%%%%%%%%%%%%%%%%%%%%%%
%%%%%%%%%%%%-----封面----%%%%%
%%%%%%%%%%%%%%%%%%%%%%%%%%%%%%
\song\sihao\thispagestyle{empty} \indent { 索取号: \ \underline{O242/6.431
}\hspace{5cm}密级:\underline{公\quad 开}}

% \thispagestyle{sty}determines characteristics of head and foot for the current page only.
% Used to override \pagestyle (q.v.) temporarily.

%\noindent suppresses indentation of first line of paragraph 没有缩进

% \hspace{len} leaves a horizontal space of dimension len .

\vspace*{8mm}
%添加校名图
\begin{center}
\includegraphics[width=85mm]{xiaoming-eps-converted-to.pdf}
\end{center}

\vspace{6mm}
\centerline{\hei\xiaoyihao\textbf{ 硕\ \ 士\ \ 学\ \ 位\ \ 论\ \ 文}} %居中,黑体,小一号,加粗
\vspace{8mm}

%添加校徽图
\begin{center}
\includegraphics[width=24mm]{xiaohui-eps-converted-to.pdf}
\end{center}

\vspace{13mm}

%标题 黑体,小一号,加粗,居中 (这次用的\begin \end)
\begin{center}
\hei\xiaoyihao\textbf{拟地球自转方程的一些结果}\\
\end{center}

\vspace{1.5cm}

\begin{tabbing} %tabbing 列表

\hspace*{3cm} \= \hspace{2.6cm} \= \kill
% \= in tabbing environment, sets a tab stop
% \kill in a\tabbing environment, deletes previous line so tabs can be set without outputting text.
% \> in tabbing environment is a forward tab.
% 这次的居中 用的 \centering ,注意三次的区别。
\>{\song\sihao\textbf {研\hspace{0.2cm}究\hspace{0.3cm}生\ \ :}} \>
{\centering\song\sihao\textbf{~~~~~~~~~~******~~~~~~~~~~~~~}}\\
% 总长3.4cm
\\
\>{\song\sihao\textbf {指导教师 \ \ :}}\> {\centering\song\sihao\textbf{~~~~~~~~**********~~~~~~~~~~~}} \\
\\
\>{\song\sihao\textbf {培养单位 \ \ :}}\> {\centering\song\sihao\textbf{~~~~~~~************~~~~~~~~~~}}\\
\\
\>{\song\sihao\textbf {一级学科 \ \ :}}\> {\centering\song\sihao\textbf{~~~~~~~~~~~*****~~~~~~~~~~~~}}\\
\\
\>{\song\sihao\textbf {二级学科 \ \ :}}\> {\centering\song\sihao\textbf{~~*******************~~~~~~~}} \\
\\
\>{\song\sihao\textbf {完成时间 \ \ :}}\> {\centering\song\sihao\textbf{~~~~~~2013年5月12日~~~~~~~~~~}}\\
\\
\>{\song\sihao\textbf {答辩时间 \ \ :}}\> {\centering\song\sihao\textbf{~~~~~~2013年5月26日~~~~~~~~~~}} \\

\end{tabbing}

\end{CJK*}
\end{document}